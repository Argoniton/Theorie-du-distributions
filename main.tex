% !TEX encoding = UTF-8 Unicode
% !TEX TS-program = xelatex

Walter Aschbacher

Table des matières

	\begin{itemize}
		\item Espaces de fonctionnes tests
		\item Distributions
		\item Opérateurs élémentaires sur les distributions
		\item Convolution
		\item Solution fondamentales
		\item Discute la distribution tempérées
	\end{itemize}
	
cm 1 1' 2 2'...;
TD; CC, CT 
Théorie du distributions. (12 * 2 + 1)h CM+TD 




\chapter{Espaces de fonctions tests} % (fold)
\label{cha:espaces_de_fonctions_tests}

\section{Espaces vectoriels topologiques localement convexes et séparés} % (fold)
\label{sec:espaces_de_fonctions_tests}
Pas de profondes.

La topologie d'un espace vectoriel complexe $V$ n'est pas toujours donnée par une norme comme dans le d'un cas espace de Banach ou d'un espace de Hilbert.
En plus, dans une topologie quelconque, les opérations linéaires de $V$, c-à-d, l'addition et la multiplication par un scalaire complexe, respectivement notées 

\[
	\mqty{V\times V\rightarrow V\\
	(x,y)\mapsto x+y},
	\mqty{\C\times V \rightarrow  V\\
	(\lambda, x)\mapsto \lambda x}
\]


ne sont pas nécessairement continues.

Alors, nous commençons section par discuter un procédé général qui nous permettra d'in une topologie sur $V$ t.q. ce opérations linéaires deviennent des applications continues p.r. à cette topologie.


Dans ce cours, sans que rien d'autre ne soit explicitement indique, $V$ sera toujours un espace vectoriel complexe.

\begin{definition}
	Soit V un espace vectoriel. Une topologie sur V telle que les opérations linéaires sont continue par rapport est dite "compatible".

	Un espace vectoriel V muni d'une topologie compatible s'appelle un "espace vectoriel topologique (EVT)".
\end{definition}

Rappelons-nous la définition d'une topologie.

\begin{definition}
	Un topologie sur une ensemble $X$ est une famille que notée souvent $T$ de parties de $X$ ayant les propriétés suivantes:\\
	(T1) $ø\in T$, $X\in T$
	(T2) soit $A_i\in T$ pour tout $i \in I$ (où $I$ est quelconque, $I$ n'est pas nécessairement dénombrable). Alors $\cup_{i\in I}A_i\in T$.
	(T3) Soit $N\in \N^*=\{1, 2, 3,...\}$ et $A_1,..., A_N\in T$. Alors, $\cap_{i=1}^NA_i\in T$.

	Le couple $(X, T)$ s'appelle un espace topologique. En plus, les éléments de $T$ s'appellent les ensembles "ouvert" et leurs complémentaires (dans $X$), les ensembles "fermes". 
\end{definition}

\begin{example}
	\begin{itemize}[(a)]
		\item Soit $X$ un ensemble et $T=\{ø,X\}$. Alors, $T$ est une topologie sur $X$ qui s'appelle la "topologie triviale".
		\item Soit $X$ un ensemble et $T=P(X)$ l'ensemble des parties de $X$. Alors, $T$ est une topologie sur $X$ qui s'appelle la "topologie discrète".
		\item L'ensemble $T=\{ \text{unions d'intervalles ouverts de } \R \}$ est une topologie sur $\R$ (la topologie usuelle).
	\end{itemize}
\end{example}

Afin de pouvoir et introduire une topologie compatible $T$ sur un espace vectoriel $V$ pour que $(V, T)$ devienne un espace vectoriel topologique nous utiliserons la notion fondamentale suivante.

\begin{definition} %1.4
	Une "semi-norme" sur un espace vectoriel V est une application
		\[p:V\rightarrow  \R\]
	ayant les propriétés suivantes (pour tout $x,y\in V$ et tout $\lambda\in\C$):
	(SN1) $p(x+y)≤p(x)+p(y)$ (inégalité triangulaire)
	(SN2) $p(\lambda x)=|\lambda|p(x)$ (linéarité)
	
	La propriété (SN1) s'appelle l'"inégalité triangulaire" et (SN2) l'"homogénéité positive".
	Si en plus, $p$ a la propriété de "séparation", c-à-d 
	(SN3) $p(x)=0$, alors $x=0$; 
	la semi-norme $p$ s'appelle une "norme" sur $V$.
\end{definition}

\begin{proposition} %1.5
	Une semi-norme a les propriétés élémentaires suivantes.
	Soit $p$ une semi-norme sur l'espace vectoriel $V$. Alors:
	\begin{enumerate}[(a)]
		\item $p(0)=0$
		\item $|p(x)-p(y)|≤p(x-y)$ pour tout $x,y\in V$
		\item $p(x)≥0$ pour tout $x\in V$
	\end{enumerate}
\end{proposition}
\begin{proof}
	cf Exr. 1
\end{proof}

\begin{exercise} %1.6
	\begin{enumerate}[(a)]
		\item Soit $V=\C^d$. Alors, pour tout $i\in\{1,...,d\}$, l'application 
		\[p_i:V\rightarrow \R,\]
		définie, pour tout $x=(x_1, ..., x_d)\in \C^d$, par 
		\[p_i(x):=|x_i|\]
		est une semi-norme sur $V$.
		\item Soit $V$ un espace vectoriel et $T:V\rightarrow \C$ une forme linéaire sur $V$. Alors, l'application
		\[p:V\rightarrow \R,\]
		définie pour tout $X\in V$, par
		\[p(x):=|T(x)|,\]
		est une semi-norme sur $V$.
	\end{enumerate}
\end{exercise}
\begin{proof} % 1.6
	cf. Exr. 2
\end{proof}
\begin{remark}
	Pour décrire la topologie usuelle sur $\C^d$, il ne suffit pas d'utiliser un soul $p_i$ de Ex. 1.6(a), mais toutes les semi-normes de la famille.
	$P:=\{p_1,...,p_d\}$ sont nécessaires.
\end{remark}

A présent, nous allons introduire la notation utilisée dans ce cours.

\begin{definition} %1.7
	Dans ce cours, si rien d'autre n'est explicitement indiquée, le sous ensemble $\Omega\in \R$ est toujours un \emph{ouvert} non-vide de $\R^T$.
	Pour $a = (α_1, α_2, ..., α_d)\in \N^d$ appelé un "multi-indice", et $x=(x_1,...,x_d)\in\Omega$ on écrira:
		\[|α|:=∑_{i=1}^d α_i\]
		\[ D^{α=0}:=1\]
		\[ D^α:=\frac{\partial^{|α|}}{\partial x_1^{α_1}...x_d^{α_d}}\]
\end{definition}
\begin{example}
	Pour $d=3$ et $α=(1,0,2)$,
	on a $D^α = D^{(1,0,2)}=\pdv{{}^3}{x_1}{{}^2x_3}$
\end{example}

Soit $m\in\N$. L'espace vectoriel complexe des fonctions de $\Omega$ dans $\C$ qui sont $m$ fois continûment dérivables est noté
$C^m(\Omega)=\{\phi:\Omega \rightarrow  \C | D^α φ\in C(Ω)\text{ pour tout }α\in \N^d avec |α|≤m\}$,
et $C^0(Ω)=C(Ω)$ sont les foncions continues sur $Ω$.
Si $K\subset Ω$ est un sous-ensemble compact (c-à-d, borné et fermé) de $Ω$, on écrira $K\Subset Ω$.

Les semi-normes suivantes seront d'une grande importance par la suite.

\begin{proposition} % 1.9
	Soit $V=C^m(Ω)$ pour un $m\in \N$, soit $K\Subset Ω$ et $l\in \N$ avec $0≤l≤m$. Alors:
	\begin{enumerate}[(a)]
		\item L'application
		\[P_{K,l}:V\rightarrow \R\]
		définie, pour tout $φ\in V$ par
		\[P_{K,l}(φ):=\sup_{\substack{X\in K\\|α|≤l}}|(D^αφ)(x)|,\]
		est une semi-norme sur $V$.
		\item L'application $q_{K,l}:V\rightarrow \R$,
		définie, pour tout $φ\in V$, par
			\[q_{K,l}(φ):=\sqrt{∑_{|α|≤l}∫_K\dd{x}|(D^αφ)(x)|^2},\]
		est une semi-norme sur $V$. (n'est pas une norme)
	\end{enumerate}
\end{proposition}
\begin{proof}
	Nous avons vérifier les propriétés d'une semi-norme dans Déf. 1.4.
	\begin{enumerate}[(a)]
		\item (SN1) Pour tout $φ,ψ\in C^m(Ω)$, on a:
		
		\begin{align*}
			P_{K,l}(φ+ψ)
				&=\sup_{\substack{x\in K\\|α|≤l}}|D^α(φ+ψ)(x)|\\ 
				&=\sup_{\substack{x\in K\\|α|≤l}} | D^α(φ)(x)+\underbrace{D^α(ψ)(x)}_{\in\C} | \\
				&≤\sup_{\substack{x\in K\\|α|≤l}}(|D^α(φ)(x)|+|D^α(ψ)(x)|)\\ 
				&≤ \sup_{\substack{x\in K\\|α|≤l}}|D^α(φ)(x)|+\sup_{\substack{x\in K\\|α|≤l}}|D^α(ψ)(x)|\\
				&=P_{K,l}(φ)+P_{K,l}(ψ)
		\end{align*}

		\item (SN2) Pour tout $φ\in C^m(Ω)$ et $λ\in \C$, on a
		\[P_{K,l}(λφ)=\sup_{\substack{x\in K\\|α|≤l}}|D^α(λφ)(x)|=\sup_{\substack{x\in K\\|α|≤l}}|λ(D^αφ)(x)|=|λ|\sup_{\substack{x\in K\\|α|≤l}}|(D^αφ)(x)|=|λ|P_{K,l}(φ)\]
		\item cf. Exr. 3
	\end{enumerate}
\end{proof}

\begin{definition} % 1.10
	Soient $p_1$ et $p_2$ deux semi-normes sur $V$. La semi-norme $p_1$ est dite "plus petite" que $p_2$, noté $p_1<p_2$, si, pour tout $x\in V$, on a que
	\[p_1(x)≤p_2(x)\]
	
	Les semi-normes $p_1$ et $p_2$ s'appellent "comparables" si $p_1<p_2$ ou $p_2<p_1$.
\end{definition}

Deux semi-normes ne sont pas nécessairement comparables comme on peut constater dans la partie (a) de l'exemple suivant.

\begin{example} %1.11
	\begin{enumerate}[(a)]
		\item Les semi-normes $p_i$ et $p_j$ de Ex. 1.6(a) ne sont pas comparables si $i≠j$. ($d=2$ : $p_1(x)=|x_1|$, $p_2(x)=|x_2|$. $x=\vc{0\\1}$, $y=\vc{1\\0}$)
		\item Soient $K_1, K_2\Subset Ω$ et $K_1\subseteq K_2$, et soient $l_1, l_2\in \N$ avec $0≤l_1≤l_2≤m$. Alors, les semi-normes de Prop. 1.9 sont comparables, c-à-d, on a
		\[\mqty{P_{K_1,l_1}<P_{K_2,l_2}\\q_{K_1,l_1}<q_{K_2,l_2}}\]
	\end{enumerate}
	
\end{example}
\begin{proof} %1.11
	cf. Exr. 4
\end{proof}

Pour tout $r>0$ et toute semi-norme $p$ sur $V$, on définit la semi-norme $rp$ sur $V$ par $(rp)(x):=rp(x)$ pour tout $x\in V$. La notion suivante sera utilisé dans la condition de la topologie compatible d'EVT.

\begin{definition}
	Une famille $P$ de semi-norme sur un espace vectoriel $V$ s'appelle "filtrante" si, pour tout $p_1,p_2 \in P$, il existe $p\in P$ et $r_1,r_2>0$ t.q.
		\[r_1p_1<p,\quad r_2p_2<p\]
\end{definition}

Nous retenons les faits suivants.

\begin{proposition} %1.13
	\begin{enumerate}[(a)]
		\item Soit $P=\{p_1,...,p_N\}$ avec $N\in\N^*$ une famille finie de semi-normes sur l'espace vectoriel $V$. Alors, l'application:
			\[\bigvee_{i=1}^Np_i:V\rightarrow \R\]
		définie, pour tout $x\in V$, par 
			\[(\bigvee_{i=1}^Np_i)(x):=\max\{p_1(x),...p_N(x)\}\]
			est semi-norme sur $V$.
		
		\item Soit $P$ une famille de semi-normes sur $V$ et soit la famille $P'$ par:
			\[q\in P' :\Leftrightarrow \text{il existe}:\]
			$p_1, ...,p_N\in P et r_1,...,r_N>0$ pour $N\in\N^*$ t.q. $q=\bigvee_{i=1}^Nr_ip_i$.
			
			Alors, $P'$ est une famille filtrante de semi-normes qui contient $P$. Elle s'appelle al "complétion filtrante" de $P$.
	\end{enumerate}
\end{proposition}
\begin{proof}
	\begin{enumerate}[(a)]
		\item D'abord pour $q:=\bigvee_{i=1}^Np_i$, on a en effet que $q:V\rightarrow \R$. Ensuite, nous allons vérifier les propriétés d'une semi-norme spécifiés dans Déf. 1.4:
		(SN1) pour tout $x,y\in V$, on a
		\[q(x+y)=\max\{\underbrace{p_1(x+y)}_{q(x)+q(y)},...,p_N(x+y)\}≤q(x)+q(y)\]
		
		(SN2) Pour tout $x\in V$ et tout $λ\in \C$, on a
		\[q(λx)=\max\{\underbrace{p_1(λx)}_{|λ|p_1(x)},...,p_N(λx)\}=|λ|\max\{p_1(x),...,p_N(x)\}=q(x)\]
		
		\item ef. notes
	\end{enumerate}
\end{proof}

\begin{remark}
	Grâce à Prop. 1.13(b), nous supposerons dorénavant qu'une famille de semi-normes est filtrante.
\end{remark}

Les familles de semi-normes suivantes joueront également un rôle important plus tard.

\begin{example} %1.15
	Soient $P_{K,l}$ et $q_{K,l}$ les semi-normes de Prop. 1.9. Alors, les familles de semi-normes sont filtrantes:
	\begin{enumerate}[(a)]
		\item $P_m(Ω):=\{P_{K,l}|K\Subset Ω, 0≤l≤m\}$
		\item $Q_m(Ω):=\{q_{K,l}|K\Subset Ω, 0≤l≤m\}$
	\end{enumerate}
\end{example}
\begin{proof}
	cf. Exr. 5
\end{proof}

Après cette courte discussion sur les familles de semi-normes, nous revenons a présent à la question de la définition d'une topologie compatible sur un espace vectoriel.

\begin{definition} %1.16
	Soit $p$ une semi-norme sur l'espace vectoriel $V$. Pour tout $x\in V$ et tout $ε>0$, l'ensemble
	\[B_{p,ε}(x)=\{y\in V|p(x-y)<ε\}\]
	s'appelle la "$p$-boule (ouverte)" de centre $x$ et de "rayon" $ε$.
\end{definition}

Ces boules ont les propriétés élémentaires suivantes.

\begin{proposition} %1.17
	Soit $p$ une semi-norme sur l'espace vectoriel $V$. Alors:
	\begin{enumerate}[(a)]
		\item La $p$-boule est "invariante par translation", c-à-d, pour tout $x\in V$, on a $B_{p,ε}(x)=x+\overbrace{B_{p,ε}}^{B_{p,ε}(0)}$
		où nous avons utilisé la notation $x+B_{p,ε}:=\{x+y|y\in B_{p,ε}$.
		\item La $p$-boule est "sphérique", c-à-d, pour tout $x\in B_{p, ε}$ et tout $λ\in\C$ avec $|λ|≤1$, on a $λx\in B_{p,ε}$.
		\item La $p$-boule est "convexe", c-à-d, pour tout $x,y\in B_{p, ε}$ et tout $0≤λ≤1$, on a
			\[λx+(1-λ)y\in B_{p,ε}.\]
		\item La $p$-boule est "absorbante", c-à-d, pour tout $x\in V$, il existe $λ>0$ t.q. 
			\[λx\in B_{p,ε}.\]
		\item Soient $p_1$ et $p_2$ des semi-normes sur $V$ et soient $x_1, x_2 \in V$ et $ε_1, ε_2>0$. Alors, si $B_{p,ε_1}(x_1)\cap B_{p,ε_2}(x_2)≠ø$, il existe $x\in V$, une semi-norme $p$ sur $V$ et $ε>0$ t.q.
			\[B_{p,ε}(x)\subseteq B_{p_1,ε_2}(x_1)\cap B_{p_2,ε_2}(x_2).\]
	\end{enumerate}
\end{proposition}

\begin{proof}
	(a)-(d) cf.Exr.6\\
	(e) Soit $x\in B_{p_1,ε_1}(x_1)\cap B_{p_2,ε_2}(x_2)$ et 	$ε = \min\{ε_1-p_1(x-x_1),ε_2-p_1(x-x_2)\}$, et soit $p:=\bigvee_{i=1}^2p_i$. Alors, pour tout $y\in B_{p,ε}(x)$, on trouve:
		$p_1(y-x_1)=p_1((y-x)+(x-x_1))≤\underbrace{p_1(y-x)}_{≤p(y-x)<ε≤ε_1-p_1(x_1-x)}+p_1(x-x_1)<ε_1$, et de manière analogue, on obtient $p_2(y-x_2)<ε_2$.
\end{proof}
% section espaces_de_fonctions_tests (end) 
% chapter espaces_de_fonctions_tests (end)